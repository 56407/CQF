
% Default to the notebook output style

    


% Inherit from the specified cell style.




    
\documentclass{article}

    
    
    \usepackage{graphicx} % Used to insert images
    \usepackage{adjustbox} % Used to constrain images to a maximum size 
    \usepackage{color} % Allow colors to be defined
    \usepackage{enumerate} % Needed for markdown enumerations to work
    \usepackage{geometry} % Used to adjust the document margins
    \usepackage{amsmath} % Equations
    \usepackage{amssymb} % Equations
    \usepackage{eurosym} % defines \euro
    \usepackage[mathletters]{ucs} % Extended unicode (utf-8) support
    \usepackage[utf8x]{inputenc} % Allow utf-8 characters in the tex document
    \usepackage{fancyvrb} % verbatim replacement that allows latex
    \usepackage{grffile} % extends the file name processing of package graphics 
                         % to support a larger range 
    % The hyperref package gives us a pdf with properly built
    % internal navigation ('pdf bookmarks' for the table of contents,
    % internal cross-reference links, web links for URLs, etc.)
    \usepackage{hyperref}
    \usepackage{longtable} % longtable support required by pandoc >1.10
    \usepackage{booktabs}  % table support for pandoc > 1.12.2
    

    
    
    \definecolor{orange}{cmyk}{0,0.4,0.8,0.2}
    \definecolor{darkorange}{rgb}{.71,0.21,0.01}
    \definecolor{darkgreen}{rgb}{.12,.54,.11}
    \definecolor{myteal}{rgb}{.26, .44, .56}
    \definecolor{gray}{gray}{0.45}
    \definecolor{lightgray}{gray}{.95}
    \definecolor{mediumgray}{gray}{.8}
    \definecolor{inputbackground}{rgb}{.95, .95, .85}
    \definecolor{outputbackground}{rgb}{.95, .95, .95}
    \definecolor{traceback}{rgb}{1, .95, .95}
    % ansi colors
    \definecolor{red}{rgb}{.6,0,0}
    \definecolor{green}{rgb}{0,.65,0}
    \definecolor{brown}{rgb}{0.6,0.6,0}
    \definecolor{blue}{rgb}{0,.145,.698}
    \definecolor{purple}{rgb}{.698,.145,.698}
    \definecolor{cyan}{rgb}{0,.698,.698}
    \definecolor{lightgray}{gray}{0.5}
    
    % bright ansi colors
    \definecolor{darkgray}{gray}{0.25}
    \definecolor{lightred}{rgb}{1.0,0.39,0.28}
    \definecolor{lightgreen}{rgb}{0.48,0.99,0.0}
    \definecolor{lightblue}{rgb}{0.53,0.81,0.92}
    \definecolor{lightpurple}{rgb}{0.87,0.63,0.87}
    \definecolor{lightcyan}{rgb}{0.5,1.0,0.83}
    
    % commands and environments needed by pandoc snippets
    % extracted from the output of `pandoc -s`
    \providecommand{\tightlist}{%
      \setlength{\itemsep}{0pt}\setlength{\parskip}{0pt}}
    \DefineVerbatimEnvironment{Highlighting}{Verbatim}{commandchars=\\\{\}}
    % Add ',fontsize=\small' for more characters per line
    \newenvironment{Shaded}{}{}
    \newcommand{\KeywordTok}[1]{\textcolor[rgb]{0.00,0.44,0.13}{\textbf{{#1}}}}
    \newcommand{\DataTypeTok}[1]{\textcolor[rgb]{0.56,0.13,0.00}{{#1}}}
    \newcommand{\DecValTok}[1]{\textcolor[rgb]{0.25,0.63,0.44}{{#1}}}
    \newcommand{\BaseNTok}[1]{\textcolor[rgb]{0.25,0.63,0.44}{{#1}}}
    \newcommand{\FloatTok}[1]{\textcolor[rgb]{0.25,0.63,0.44}{{#1}}}
    \newcommand{\CharTok}[1]{\textcolor[rgb]{0.25,0.44,0.63}{{#1}}}
    \newcommand{\StringTok}[1]{\textcolor[rgb]{0.25,0.44,0.63}{{#1}}}
    \newcommand{\CommentTok}[1]{\textcolor[rgb]{0.38,0.63,0.69}{\textit{{#1}}}}
    \newcommand{\OtherTok}[1]{\textcolor[rgb]{0.00,0.44,0.13}{{#1}}}
    \newcommand{\AlertTok}[1]{\textcolor[rgb]{1.00,0.00,0.00}{\textbf{{#1}}}}
    \newcommand{\FunctionTok}[1]{\textcolor[rgb]{0.02,0.16,0.49}{{#1}}}
    \newcommand{\RegionMarkerTok}[1]{{#1}}
    \newcommand{\ErrorTok}[1]{\textcolor[rgb]{1.00,0.00,0.00}{\textbf{{#1}}}}
    \newcommand{\NormalTok}[1]{{#1}}
    
    % Define a nice break command that doesn't care if a line doesn't already
    % exist.
    \def\br{\hspace*{\fill} \\* }
    % Math Jax compatability definitions
    \def\gt{>}
    \def\lt{<}
    % Document parameters
    \title{Asian Option Pricing using Monte Carlo}
    \author{Tanya Sandoval}
    
    
    

    % Pygments definitions
    
\makeatletter
\def\PY@reset{\let\PY@it=\relax \let\PY@bf=\relax%
    \let\PY@ul=\relax \let\PY@tc=\relax%
    \let\PY@bc=\relax \let\PY@ff=\relax}
\def\PY@tok#1{\csname PY@tok@#1\endcsname}
\def\PY@toks#1+{\ifx\relax#1\empty\else%
    \PY@tok{#1}\expandafter\PY@toks\fi}
\def\PY@do#1{\PY@bc{\PY@tc{\PY@ul{%
    \PY@it{\PY@bf{\PY@ff{#1}}}}}}}
\def\PY#1#2{\PY@reset\PY@toks#1+\relax+\PY@do{#2}}

\expandafter\def\csname PY@tok@gd\endcsname{\def\PY@tc##1{\textcolor[rgb]{0.63,0.00,0.00}{##1}}}
\expandafter\def\csname PY@tok@gu\endcsname{\let\PY@bf=\textbf\def\PY@tc##1{\textcolor[rgb]{0.50,0.00,0.50}{##1}}}
\expandafter\def\csname PY@tok@gt\endcsname{\def\PY@tc##1{\textcolor[rgb]{0.00,0.27,0.87}{##1}}}
\expandafter\def\csname PY@tok@gs\endcsname{\let\PY@bf=\textbf}
\expandafter\def\csname PY@tok@gr\endcsname{\def\PY@tc##1{\textcolor[rgb]{1.00,0.00,0.00}{##1}}}
\expandafter\def\csname PY@tok@cm\endcsname{\let\PY@it=\textit\def\PY@tc##1{\textcolor[rgb]{0.25,0.50,0.50}{##1}}}
\expandafter\def\csname PY@tok@vg\endcsname{\def\PY@tc##1{\textcolor[rgb]{0.10,0.09,0.49}{##1}}}
\expandafter\def\csname PY@tok@m\endcsname{\def\PY@tc##1{\textcolor[rgb]{0.40,0.40,0.40}{##1}}}
\expandafter\def\csname PY@tok@mh\endcsname{\def\PY@tc##1{\textcolor[rgb]{0.40,0.40,0.40}{##1}}}
\expandafter\def\csname PY@tok@go\endcsname{\def\PY@tc##1{\textcolor[rgb]{0.53,0.53,0.53}{##1}}}
\expandafter\def\csname PY@tok@ge\endcsname{\let\PY@it=\textit}
\expandafter\def\csname PY@tok@vc\endcsname{\def\PY@tc##1{\textcolor[rgb]{0.10,0.09,0.49}{##1}}}
\expandafter\def\csname PY@tok@il\endcsname{\def\PY@tc##1{\textcolor[rgb]{0.40,0.40,0.40}{##1}}}
\expandafter\def\csname PY@tok@cs\endcsname{\let\PY@it=\textit\def\PY@tc##1{\textcolor[rgb]{0.25,0.50,0.50}{##1}}}
\expandafter\def\csname PY@tok@cp\endcsname{\def\PY@tc##1{\textcolor[rgb]{0.74,0.48,0.00}{##1}}}
\expandafter\def\csname PY@tok@gi\endcsname{\def\PY@tc##1{\textcolor[rgb]{0.00,0.63,0.00}{##1}}}
\expandafter\def\csname PY@tok@gh\endcsname{\let\PY@bf=\textbf\def\PY@tc##1{\textcolor[rgb]{0.00,0.00,0.50}{##1}}}
\expandafter\def\csname PY@tok@ni\endcsname{\let\PY@bf=\textbf\def\PY@tc##1{\textcolor[rgb]{0.60,0.60,0.60}{##1}}}
\expandafter\def\csname PY@tok@nl\endcsname{\def\PY@tc##1{\textcolor[rgb]{0.63,0.63,0.00}{##1}}}
\expandafter\def\csname PY@tok@nn\endcsname{\let\PY@bf=\textbf\def\PY@tc##1{\textcolor[rgb]{0.00,0.00,1.00}{##1}}}
\expandafter\def\csname PY@tok@no\endcsname{\def\PY@tc##1{\textcolor[rgb]{0.53,0.00,0.00}{##1}}}
\expandafter\def\csname PY@tok@na\endcsname{\def\PY@tc##1{\textcolor[rgb]{0.49,0.56,0.16}{##1}}}
\expandafter\def\csname PY@tok@nb\endcsname{\def\PY@tc##1{\textcolor[rgb]{0.00,0.50,0.00}{##1}}}
\expandafter\def\csname PY@tok@nc\endcsname{\let\PY@bf=\textbf\def\PY@tc##1{\textcolor[rgb]{0.00,0.00,1.00}{##1}}}
\expandafter\def\csname PY@tok@nd\endcsname{\def\PY@tc##1{\textcolor[rgb]{0.67,0.13,1.00}{##1}}}
\expandafter\def\csname PY@tok@ne\endcsname{\let\PY@bf=\textbf\def\PY@tc##1{\textcolor[rgb]{0.82,0.25,0.23}{##1}}}
\expandafter\def\csname PY@tok@nf\endcsname{\def\PY@tc##1{\textcolor[rgb]{0.00,0.00,1.00}{##1}}}
\expandafter\def\csname PY@tok@si\endcsname{\let\PY@bf=\textbf\def\PY@tc##1{\textcolor[rgb]{0.73,0.40,0.53}{##1}}}
\expandafter\def\csname PY@tok@s2\endcsname{\def\PY@tc##1{\textcolor[rgb]{0.73,0.13,0.13}{##1}}}
\expandafter\def\csname PY@tok@vi\endcsname{\def\PY@tc##1{\textcolor[rgb]{0.10,0.09,0.49}{##1}}}
\expandafter\def\csname PY@tok@nt\endcsname{\let\PY@bf=\textbf\def\PY@tc##1{\textcolor[rgb]{0.00,0.50,0.00}{##1}}}
\expandafter\def\csname PY@tok@nv\endcsname{\def\PY@tc##1{\textcolor[rgb]{0.10,0.09,0.49}{##1}}}
\expandafter\def\csname PY@tok@s1\endcsname{\def\PY@tc##1{\textcolor[rgb]{0.73,0.13,0.13}{##1}}}
\expandafter\def\csname PY@tok@kd\endcsname{\let\PY@bf=\textbf\def\PY@tc##1{\textcolor[rgb]{0.00,0.50,0.00}{##1}}}
\expandafter\def\csname PY@tok@sh\endcsname{\def\PY@tc##1{\textcolor[rgb]{0.73,0.13,0.13}{##1}}}
\expandafter\def\csname PY@tok@sc\endcsname{\def\PY@tc##1{\textcolor[rgb]{0.73,0.13,0.13}{##1}}}
\expandafter\def\csname PY@tok@sx\endcsname{\def\PY@tc##1{\textcolor[rgb]{0.00,0.50,0.00}{##1}}}
\expandafter\def\csname PY@tok@bp\endcsname{\def\PY@tc##1{\textcolor[rgb]{0.00,0.50,0.00}{##1}}}
\expandafter\def\csname PY@tok@c1\endcsname{\let\PY@it=\textit\def\PY@tc##1{\textcolor[rgb]{0.25,0.50,0.50}{##1}}}
\expandafter\def\csname PY@tok@kc\endcsname{\let\PY@bf=\textbf\def\PY@tc##1{\textcolor[rgb]{0.00,0.50,0.00}{##1}}}
\expandafter\def\csname PY@tok@c\endcsname{\let\PY@it=\textit\def\PY@tc##1{\textcolor[rgb]{0.25,0.50,0.50}{##1}}}
\expandafter\def\csname PY@tok@mf\endcsname{\def\PY@tc##1{\textcolor[rgb]{0.40,0.40,0.40}{##1}}}
\expandafter\def\csname PY@tok@err\endcsname{\def\PY@bc##1{\setlength{\fboxsep}{0pt}\fcolorbox[rgb]{1.00,0.00,0.00}{1,1,1}{\strut ##1}}}
\expandafter\def\csname PY@tok@mb\endcsname{\def\PY@tc##1{\textcolor[rgb]{0.40,0.40,0.40}{##1}}}
\expandafter\def\csname PY@tok@ss\endcsname{\def\PY@tc##1{\textcolor[rgb]{0.10,0.09,0.49}{##1}}}
\expandafter\def\csname PY@tok@sr\endcsname{\def\PY@tc##1{\textcolor[rgb]{0.73,0.40,0.53}{##1}}}
\expandafter\def\csname PY@tok@mo\endcsname{\def\PY@tc##1{\textcolor[rgb]{0.40,0.40,0.40}{##1}}}
\expandafter\def\csname PY@tok@kn\endcsname{\let\PY@bf=\textbf\def\PY@tc##1{\textcolor[rgb]{0.00,0.50,0.00}{##1}}}
\expandafter\def\csname PY@tok@mi\endcsname{\def\PY@tc##1{\textcolor[rgb]{0.40,0.40,0.40}{##1}}}
\expandafter\def\csname PY@tok@gp\endcsname{\let\PY@bf=\textbf\def\PY@tc##1{\textcolor[rgb]{0.00,0.00,0.50}{##1}}}
\expandafter\def\csname PY@tok@o\endcsname{\def\PY@tc##1{\textcolor[rgb]{0.40,0.40,0.40}{##1}}}
\expandafter\def\csname PY@tok@kr\endcsname{\let\PY@bf=\textbf\def\PY@tc##1{\textcolor[rgb]{0.00,0.50,0.00}{##1}}}
\expandafter\def\csname PY@tok@s\endcsname{\def\PY@tc##1{\textcolor[rgb]{0.73,0.13,0.13}{##1}}}
\expandafter\def\csname PY@tok@kp\endcsname{\def\PY@tc##1{\textcolor[rgb]{0.00,0.50,0.00}{##1}}}
\expandafter\def\csname PY@tok@w\endcsname{\def\PY@tc##1{\textcolor[rgb]{0.73,0.73,0.73}{##1}}}
\expandafter\def\csname PY@tok@kt\endcsname{\def\PY@tc##1{\textcolor[rgb]{0.69,0.00,0.25}{##1}}}
\expandafter\def\csname PY@tok@ow\endcsname{\let\PY@bf=\textbf\def\PY@tc##1{\textcolor[rgb]{0.67,0.13,1.00}{##1}}}
\expandafter\def\csname PY@tok@sb\endcsname{\def\PY@tc##1{\textcolor[rgb]{0.73,0.13,0.13}{##1}}}
\expandafter\def\csname PY@tok@k\endcsname{\let\PY@bf=\textbf\def\PY@tc##1{\textcolor[rgb]{0.00,0.50,0.00}{##1}}}
\expandafter\def\csname PY@tok@se\endcsname{\let\PY@bf=\textbf\def\PY@tc##1{\textcolor[rgb]{0.73,0.40,0.13}{##1}}}
\expandafter\def\csname PY@tok@sd\endcsname{\let\PY@it=\textit\def\PY@tc##1{\textcolor[rgb]{0.73,0.13,0.13}{##1}}}

\def\PYZbs{\char`\\}
\def\PYZus{\char`\_}
\def\PYZob{\char`\{}
\def\PYZcb{\char`\}}
\def\PYZca{\char`\^}
\def\PYZam{\char`\&}
\def\PYZlt{\char`\<}
\def\PYZgt{\char`\>}
\def\PYZsh{\char`\#}
\def\PYZpc{\char`\%}
\def\PYZdl{\char`\$}
\def\PYZhy{\char`\-}
\def\PYZsq{\char`\'}
\def\PYZdq{\char`\"}
\def\PYZti{\char`\~}
% for compatibility with earlier versions
\def\PYZat{@}
\def\PYZlb{[}
\def\PYZrb{]}
\makeatother


    % Exact colors from NB
    \definecolor{incolor}{rgb}{0.0, 0.0, 0.5}
    \definecolor{outcolor}{rgb}{0.545, 0.0, 0.0}



    
    % Prevent overflowing lines due to hard-to-break entities
    \sloppy 
    % Setup hyperref package
    \hypersetup{
      breaklinks=true,  % so long urls are correctly broken across lines
      colorlinks=true,
      urlcolor=blue,
      linkcolor=darkorange,
      citecolor=darkgreen,
      }
    % Slightly bigger margins than the latex defaults
    
    \geometry{verbose,tmargin=0.8in,bmargin=0.8in,lmargin=1in,rmargin=1in}
    
    

    \begin{document}
    
    
    \maketitle
    
    

    
    \section{Asian Option Pricing using Monte
Carlo}\label{asian-option-pricing-using-monte-carlo}

This report presents the results of using Monte Carlo (MC) to price an
\textbf{Asian Call option} as given by:

\begin{equation}
V(S, t) = \mathbb{E}^\mathbb{Q} [e^{-\int^T_t rdr} \mathrm{Payoff}(S_T)] 
\end{equation}

For brevity, the results for a put option have been omitted as well as the details of the theory and mathematics of Asian options. The documentation of the {\it{Milstein}} convention and {\it antithetic variance reduction} technique can also be found elsewhere. This model was developed using {\textbf{Python 2.7}} and all the related code can be found as a separate attachment. 

    \subsection{Initial parameters}\label{initial-parameters}

The following parameters are used for the simulation, unless otherwise
stated:

\begin{itemize}
\tightlist
\item
  Today's stock price $S_0$ = 100
\item
  Strike $K$ = 100
\item
  Time to expiry $(T-t)$ = 1 year (100 time steps)
\item
  Volatility $\sigma$= 20\%
\item
  Constant risk-free interest rate $r$ = 5\%
\end{itemize}

%    \begin{Verbatim}[commandchars=\\\{\}]
%{\color{incolor}In [{\color{incolor}1}]:} \PY{o}{\PYZpc{}}\PY{k}{matplotlib} inline
%        
%        \PY{o}{\PYZpc{}}\PY{k}{run} ../SAsianOption.py
%        \PY{c}{\PYZsh{} Above line runs the following function }
%        \PY{c}{\PYZsh{} dic = asian\PYZus{}option\PYZus{}simulator(S0=100., K=100., T=1.0, r=0.05, sigma=0.2, M=100, I=100, k=10)}
%        
%        \PY{c}{\PYZsh{} Sets plot size in notebook}
%        \PY{k+kn}{import} \PY{n+nn}{matplotlib.pylab} \PY{k+kn}{as} \PY{n+nn}{pylab}
%        \PY{n}{pylab}\PY{o}{.}\PY{n}{rcParams}\PY{p}{[}\PY{l+s}{\PYZsq{}}\PY{l+s}{figure.figsize}\PY{l+s}{\PYZsq{}}\PY{p}{]} \PY{o}{=} \PY{p}{(}\PY{l+m+mi}{10}\PY{p}{,} \PY{l+m+mi}{6}\PY{p}{)}
%\end{Verbatim}
%
%    \begin{Verbatim}[commandchars=\\\{\}]
%EU V(T) = 10.0892013234
%Asian AC\_c(T) = 5.54659143575
%Asian AC\_d(T) = 5.49974029651
%Asian GC\_c(T) = 5.33383690461
%    \end{Verbatim}

    \subsection{Stock Path Generation}\label{stock-path-generation}

We start by using MC to generate 100 stock price paths using the
\emph{Milstein} convention for a risk-neutral random walk, which makes use of Ito's lemma to increase the accuracy of the approximation by adding a second order term. In this process we use the \emph{antithetic
variance reduction technique} which increases the statistics to compute
the option price at minimal computational effort. The below plot shows 100 of these paths:

%    \begin{Verbatim}[commandchars=\\\{\}]
%{\color{incolor}In [{\color{incolor}2}]:} \PY{n}{plt}\PY{o}{.}\PY{n}{plot}\PY{p}{(}\PY{n}{t\PYZus{}index}\PY{p}{,} \PY{n}{S\PYZus{}plus}\PY{p}{)}
%        \PY{n}{plt}\PY{o}{.}\PY{n}{ylabel}\PY{p}{(}\PY{l+s}{\PYZdq{}}\PY{l+s}{S(t)}\PY{l+s}{\PYZdq{}}\PY{p}{)}
%        \PY{n}{plt}\PY{o}{.}\PY{n}{xlabel}\PY{p}{(}\PY{l+s}{\PYZdq{}}\PY{l+s}{t}\PY{l+s}{\PYZdq{}}\PY{p}{)}
%\end{Verbatim}
%
%            \begin{Verbatim}[commandchars=\\\{\}]
%{\color{outcolor}Out[{\color{outcolor}2}]:} <matplotlib.text.Text at 0x8e847b8>
%\end{Verbatim}
%        
    \begin{center}
    \adjustimage{max size={0.75\linewidth}{0.9\paperheight}}{asianOptionPricing_files/asianOptionPricing_4_1.png}
    \end{center}

    
    \subsection{European Call Pricing}\label{european-call-pricing}

The same methodology is used to price a European Call option in order to
compare to the Asian results later on. For the same input parameters
above the following result is obtained (later one we discuss the error in the estimate):

%    \begin{Verbatim}[commandchars=\\\{\}]
%{\color{incolor}In [{\color{incolor}3}]:} \PY{k}{print} \PY{l+s}{\PYZsq{}}\PY{l+s}{EU V(T) = \PYZob{}0\PYZcb{}}\PY{l+s}{\PYZsq{}}\PY{o}{.}\PY{n}{format}\PY{p}{(}\PY{n}{V}\PY{p}{)}
%\end{Verbatim}

    \begin{Verbatim}[commandchars=\\\{\}]
EU V(T) = 10.089201 (0.131768)
    \end{Verbatim}

    The corresponding Payoff function at expiry (t=T) is shown below, where the effect
of the variance reduction technique is shown to boost the statistics. This hence reduces the estimate error at little computational cost.

%    \begin{Verbatim}[commandchars=\\\{\}]
%{\color{incolor}In [{\color{incolor}4}]:} \PY{c}{\PYZsh{} V vs S plot without and with variance reduction}
%        \PY{n}{sns}\PY{o}{.}\PY{n}{regplot}\PY{p}{(}\PY{n}{x}\PY{o}{=}\PY{n}{S\PYZus{}join}\PY{p}{[}\PY{o}{\PYZhy{}}\PY{l+m+mi}{1}\PY{p}{]}\PY{p}{,} \PY{n}{y}\PY{o}{=}\PY{n}{V\PYZus{}join}\PY{p}{[}\PY{o}{\PYZhy{}}\PY{l+m+mi}{1}\PY{p}{]}\PY{p}{,} \PY{n}{fit\PYZus{}reg}\PY{o}{=}\PY{n+nb+bp}{False}\PY{p}{)} \PY{c}{\PYZsh{} with antithetic correction}
%        \PY{n}{sns}\PY{o}{.}\PY{n}{regplot}\PY{p}{(}\PY{n}{x}\PY{o}{=}\PY{n}{S\PYZus{}plus}\PY{p}{[}\PY{o}{\PYZhy{}}\PY{l+m+mi}{1}\PY{p}{]}\PY{p}{,} \PY{n}{y}\PY{o}{=}\PY{n}{V\PYZus{}plus}\PY{p}{[}\PY{o}{\PYZhy{}}\PY{l+m+mi}{1}\PY{p}{]}\PY{p}{,} \PY{n}{fit\PYZus{}reg}\PY{o}{=}\PY{n+nb+bp}{False}\PY{p}{,} \PY{n}{color}\PY{o}{=}\PY{l+s}{\PYZsq{}}\PY{l+s}{yellow}\PY{l+s}{\PYZsq{}}\PY{p}{,} \PY{n}{marker}\PY{o}{=}\PY{l+s}{\PYZsq{}}\PY{l+s}{+}\PY{l+s}{\PYZsq{}}\PY{p}{)} \PY{c}{\PYZsh{} without antithetic correction}
%        \PY{n}{plt}\PY{o}{.}\PY{n}{ylabel}\PY{p}{(}\PY{l+s}{\PYZsq{}}\PY{l+s}{Discounted Payoff (t=T)}\PY{l+s}{\PYZsq{}}\PY{p}{)}
%        \PY{n}{plt}\PY{o}{.}\PY{n}{xlabel}\PY{p}{(}\PY{l+s}{\PYZsq{}}\PY{l+s}{S(t=T)}\PY{l+s}{\PYZsq{}}\PY{p}{)}
%        \PY{n}{plt}\PY{o}{.}\PY{n}{legend}\PY{p}{(}\PY{p}{[}\PY{l+s}{\PYZsq{}}\PY{l+s}{With Antith. Red.}\PY{l+s}{\PYZsq{}}\PY{p}{,} \PY{l+s}{\PYZsq{}}\PY{l+s}{Without Antith. Red.}\PY{l+s}{\PYZsq{}}\PY{p}{]}\PY{p}{,} \PY{n}{loc}\PY{o}{=}\PY{l+s}{\PYZsq{}}\PY{l+s}{upper left}\PY{l+s}{\PYZsq{}}\PY{p}{,} \PY{n}{fontsize}\PY{o}{=}\PY{l+s}{\PYZsq{}}\PY{l+s}{large}\PY{l+s}{\PYZsq{}}\PY{p}{)}
%\end{Verbatim}

%            \begin{Verbatim}[commandchars=\\\{\}]
%{\color{outcolor}Out[{\color{outcolor}4}]:} <matplotlib.legend.Legend at 0x95d14e0>
%\end{Verbatim}
        
    \begin{center}
    \adjustimage{max size={0.85\linewidth}{0.9\paperheight}}{asianOptionPricing_files/asianOptionPricing_8_1.png}
    \end{center}

    
    \subsection{Asian Call Pricing}\label{asian-call-pricing}

An Asian option (or average value option) falls in the {\it{exotics}} options class. For Asian options the payoff is determined by the average of the underlying price over some pre-set period of time, hence it is a path-dependent instrument.

An Asian Call value is then calculated using the same parameters above
but with different methodologies to calculate the average of the stock
price:

\begin{itemize}
\tightlist
\item
  \textbf{Arithmetic} averaging using \textbf{both continuous} (`AC\_c')
  and \textbf{discrete} (`AC\_d') sampling
\item
  \textbf{Geometric} averaging using \textbf{continuous} sampling
  \textbf{only} (`GC\_c')
\end{itemize}

Note in the context of MC, we refer to `continuous sampling' when all the
data points available up to that time are used, whereas `discrete
sampling' uses a sample determined by sampling period $k$ (10 days in
this case).

For all the examples below 100 time steps of size 0.01 each are used to reach the option maturity at $t=T=1$, starting from $t=0$. Similarly, a \textbf{fixed strike} is implemented except in the last section. This is defined as
\begin{equation}
C(T) = \text{max}\left( A(0,T) - K, 0 \right)
\end{equation}


    \subsubsection{Continuous vs Discrete
Averaging Error}\label{continuous-vs-discrete-sampling}

First let's understand the effects of continuous vs discrete averaging.
For this purpose we restrict to the Arithmetic case only and assume the
Geometric case observes the same behaviour.

Plotting one of the simulated stock paths and corresponding arithmetic
continuous and discrete averages (using a sampling period of 10
days), we get the following:

%    \begin{Verbatim}[commandchars=\\\{\}]
%{\color{incolor}In [{\color{incolor}5}]:} \PY{c}{\PYZsh{} S plot for MC, Arithmetic cont. and discrete averages}
%        \PY{n}{plt}\PY{o}{.}\PY{n}{plot}\PY{p}{(}\PY{n}{t\PYZus{}index}\PY{p}{,} \PY{n}{S\PYZus{}join}\PY{p}{[}\PY{l+m+mi}{0}\PY{p}{:}\PY{p}{,}\PY{l+m+mi}{0}\PY{p}{:}\PY{l+m+mi}{1}\PY{p}{]}\PY{p}{,} \PY{n}{label}\PY{o}{=}\PY{l+s}{\PYZsq{}}\PY{l+s}{S\PYZus{}sim1}\PY{l+s}{\PYZsq{}}\PY{p}{)}
%        \PY{n}{plt}\PY{o}{.}\PY{n}{plot}\PY{p}{(}\PY{n}{t\PYZus{}index}\PY{p}{,} \PY{n}{A\PYZus{}c\PYZus{}join}\PY{p}{[}\PY{l+m+mi}{0}\PY{p}{:}\PY{p}{,}\PY{l+m+mi}{0}\PY{p}{:}\PY{l+m+mi}{1}\PY{p}{]}\PY{p}{,} \PY{n}{label}\PY{o}{=}\PY{l+s}{\PYZsq{}}\PY{l+s}{Cont. Avg.}\PY{l+s}{\PYZsq{}}\PY{p}{)}
%        \PY{n}{plt}\PY{o}{.}\PY{n}{plot}\PY{p}{(}\PY{n}{t\PYZus{}index}\PY{p}{,} \PY{n}{A\PYZus{}d\PYZus{}join}\PY{p}{[}\PY{l+m+mi}{0}\PY{p}{:}\PY{p}{,}\PY{l+m+mi}{0}\PY{p}{:}\PY{l+m+mi}{1}\PY{p}{]}\PY{p}{,} \PY{n}{label}\PY{o}{=}\PY{l+s}{\PYZsq{}}\PY{l+s}{Disc. Avg.}\PY{l+s}{\PYZsq{}}\PY{p}{)}
%        \PY{n}{plt}\PY{o}{.}\PY{n}{ylabel}\PY{p}{(}\PY{l+s}{\PYZsq{}}\PY{l+s}{S(t)}\PY{l+s}{\PYZsq{}}\PY{p}{)}
%        \PY{n}{plt}\PY{o}{.}\PY{n}{xlabel}\PY{p}{(}\PY{l+s}{\PYZsq{}}\PY{l+s}{t}\PY{l+s}{\PYZsq{}}\PY{p}{)}
%        \PY{n}{plt}\PY{o}{.}\PY{n}{legend}\PY{p}{(}\PY{p}{)}
%\end{Verbatim}
%
%            \begin{Verbatim}[commandchars=\\\{\}]
%{\color{outcolor}Out[{\color{outcolor}5}]:} <matplotlib.legend.Legend at 0x992bc50>
%\end{Verbatim}
%        
    \begin{center}
    \adjustimage{max size={0.85\linewidth}{0.9\paperheight}}{asianOptionPricing_files/asianOptionPricing_11_1.png}
    \end{center}
    { \hspace*{\fill} \\}
    
    Hence we note that $A_c$ and $A_d$ differ slightly. This difference grows
with the sampling period \emph{k} because for the same time to expiry
we have less data points to calculate the average. However, as the number of data points increases, we expect the difference to get smaller. This behaviour is
shown in the plot below where $d_{10}$ is more unstable than $d_4$ initially, but then both tend  to converge towards zero. This is also an example of the law of large numbers (LLN). We hence expect the error from this to be small at expiry.

%    \begin{Verbatim}[commandchars=\\\{\}]
%{\color{incolor}In [{\color{incolor}6}]:} \PY{c}{\PYZsh{} Changing discrete sampling period from k=10 to k=4}
%        \PY{n}{dic2} \PY{o}{=} \PY{n}{asian\PYZus{}option\PYZus{}simulator}\PY{p}{(}\PY{n}{S0}\PY{o}{=}\PY{l+m+mf}{100.}\PY{p}{,} \PY{n}{K}\PY{o}{=}\PY{l+m+mf}{100.}\PY{p}{,} \PY{n}{T}\PY{o}{=}\PY{l+m+mf}{1.0}\PY{p}{,} \PY{n}{r}\PY{o}{=}\PY{l+m+mf}{0.05}\PY{p}{,} \PY{n}{sigma}\PY{o}{=}\PY{l+m+mf}{0.2}\PY{p}{,} \PY{n}{M}\PY{o}{=}\PY{l+m+mi}{100}\PY{p}{,} \PY{n}{I}\PY{o}{=}\PY{l+m+mi}{100}\PY{p}{,} \PY{n}{k}\PY{o}{=}\PY{l+m+mi}{4}\PY{p}{)}
%        
%        \PY{c}{\PYZsh{} Calculate differences wrt continous case}
%        \PY{n}{d\PYZus{}10} \PY{o}{=} \PY{n}{A\PYZus{}c\PYZus{}join} \PY{o}{\PYZhy{}} \PY{n}{A\PYZus{}d\PYZus{}join}
%        \PY{n}{d\PYZus{}4} \PY{o}{=} \PY{n}{A\PYZus{}c\PYZus{}join} \PY{o}{\PYZhy{}} \PY{n}{dic2}\PY{p}{[}\PY{l+s}{\PYZsq{}}\PY{l+s}{A\PYZus{}d\PYZus{}join}\PY{l+s}{\PYZsq{}}\PY{p}{]}
%        
%        \PY{c}{\PYZsh{} Plot for just one simulation}
%        \PY{n}{plt}\PY{o}{.}\PY{n}{plot}\PY{p}{(}\PY{n}{t\PYZus{}index}\PY{p}{,} \PY{n}{d\PYZus{}4}\PY{p}{[}\PY{l+m+mi}{0}\PY{p}{:}\PY{p}{,}\PY{l+m+mi}{0}\PY{p}{:}\PY{l+m+mi}{1}\PY{p}{]}\PY{p}{,} \PY{n}{label}\PY{o}{=}\PY{l+s}{\PYZsq{}}\PY{l+s}{Disc. samp. freq=4}\PY{l+s}{\PYZsq{}}\PY{p}{)}
%        \PY{n}{plt}\PY{o}{.}\PY{n}{plot}\PY{p}{(}\PY{n}{t\PYZus{}index}\PY{p}{,} \PY{n}{d\PYZus{}10}\PY{p}{[}\PY{l+m+mi}{0}\PY{p}{:}\PY{p}{,}\PY{l+m+mi}{0}\PY{p}{:}\PY{l+m+mi}{1}\PY{p}{]}\PY{p}{,} \PY{n}{label}\PY{o}{=}\PY{l+s}{\PYZsq{}}\PY{l+s}{Disc. samp freq=10}\PY{l+s}{\PYZsq{}}\PY{p}{)}
%        \PY{n}{plt}\PY{o}{.}\PY{n}{ylabel}\PY{p}{(}\PY{l+s}{\PYZsq{}}\PY{l+s}{Diff wrt to Cont. Avg.}\PY{l+s}{\PYZsq{}}\PY{p}{)}
%        \PY{n}{plt}\PY{o}{.}\PY{n}{xlabel}\PY{p}{(}\PY{l+s}{\PYZsq{}}\PY{l+s}{t}\PY{l+s}{\PYZsq{}}\PY{p}{)}
%        \PY{n}{plt}\PY{o}{.}\PY{n}{legend}\PY{p}{(}\PY{p}{)}
%\end{Verbatim}
%
%            \begin{Verbatim}[commandchars=\\\{\}]
%{\color{outcolor}Out[{\color{outcolor}6}]:} <matplotlib.legend.Legend at 0xa981e80>
%\end{Verbatim}
%        
    \begin{center}
    \adjustimage{max size={0.85\linewidth}{0.9\paperheight}}{asianOptionPricing_files/asianOptionPricing_13_1.png}
    \end{center}

    
    However, for practical and legal reasons path-dependent quantities are
never measured continuously. There is minimum time step between sampling
since it is difficult to incorporate every single traded price into an
average, for example, data can be unreliable and the exact time of a
trade may not be known accurately.

\subsubsection{Time Step Error}\label{time-step-error}

Because we have used the \emph{Milstein} scheme, the error in the estimate from this is $O(\delta t)$ (which is better than with the \emph{Euler-Mayurama} method which is $O(\delta t^{1/2})$).
The option price  evolution with the $n$th time step to expiry is shown in
the below plot for different number of time steps. Again we see at expiry the difference between continuous and discrete averaging is minimal, but the error from the number of time steps can become quite significant. 

%    \begin{Verbatim}[commandchars=\\\{\}]
%{\color{incolor}In [{\color{incolor}7}]:} \PY{c}{\PYZsh{} Evolution of Asian Call value with number of time steps}
%        \PY{n}{c} \PY{o}{=} \PY{n}{AC\PYZus{}c\PYZus{}join}\PY{o}{.}\PY{n}{mean}\PY{p}{(}\PY{n}{axis}\PY{o}{=}\PY{l+m+mi}{1}\PY{p}{)}
%        \PY{n}{d} \PY{o}{=} \PY{n}{AC\PYZus{}d\PYZus{}join}\PY{o}{.}\PY{n}{mean}\PY{p}{(}\PY{n}{axis}\PY{o}{=}\PY{l+m+mi}{1}\PY{p}{)}
%        \PY{n}{plt}\PY{o}{.}\PY{n}{plot}\PY{p}{(}\PY{n}{t\PYZus{}index}\PY{p}{,} \PY{n}{c}\PY{p}{,} \PY{n}{label}\PY{o}{=}\PY{l+s}{\PYZsq{}}\PY{l+s}{Arithmetic Continous Avg.}\PY{l+s}{\PYZsq{}}\PY{p}{)}
%        \PY{n}{plt}\PY{o}{.}\PY{n}{plot}\PY{p}{(}\PY{n}{t\PYZus{}index}\PY{p}{,} \PY{n}{d}\PY{p}{,} \PY{n}{label}\PY{o}{=}\PY{l+s}{\PYZsq{}}\PY{l+s}{Arithmetic Discrete Avg.}\PY{l+s}{\PYZsq{}}\PY{p}{)}
%        \PY{n}{plt}\PY{o}{.}\PY{n}{ylabel}\PY{p}{(}\PY{l+s}{\PYZsq{}}\PY{l+s}{C(t)}\PY{l+s}{\PYZsq{}}\PY{p}{)}
%        \PY{n}{plt}\PY{o}{.}\PY{n}{xlabel}\PY{p}{(}\PY{l+s}{\PYZsq{}}\PY{l+s}{t}\PY{l+s}{\PYZsq{}}\PY{p}{)}
%        \PY{n}{plt}\PY{o}{.}\PY{n}{legend}\PY{p}{(}\PY{n}{loc}\PY{o}{=}\PY{l+s}{\PYZsq{}}\PY{l+s}{upper left}\PY{l+s}{\PYZsq{}}\PY{p}{,} \PY{n}{fontsize}\PY{o}{=}\PY{l+s}{\PYZsq{}}\PY{l+s}{large}\PY{l+s}{\PYZsq{}}\PY{p}{)}
%\end{Verbatim}
%
%            \begin{Verbatim}[commandchars=\\\{\}]
%{\color{outcolor}Out[{\color{outcolor}7}]:} <matplotlib.legend.Legend at 0xac0be48>
%\end{Verbatim}
        
    \begin{center}
    \adjustimage{max size={0.85\linewidth}{0.9\paperheight}}{asianOptionPricing_files/asianOptionPricing_15_1.png}
    \end{center}
    
    \subsubsection{Payoff function}\label{payoff-function}

The Payoff function at expiry $t=T$ of the Asian call is also a lot more complex
than its European equivalent, which is expected given its
path-dependent nature. This justifies the need for using Monte Carlo
methods for the pricing.

%    \begin{Verbatim}[commandchars=\\\{\}]
%{\color{incolor}In [{\color{incolor}8}]:} \PY{c}{\PYZsh{} \PYZsh{} V vs S plot}
%        \PY{n}{sns}\PY{o}{.}\PY{n}{regplot}\PY{p}{(}\PY{n}{x}\PY{o}{=}\PY{n}{S\PYZus{}join}\PY{p}{[}\PY{o}{\PYZhy{}}\PY{l+m+mi}{1}\PY{p}{]}\PY{p}{,} \PY{n}{y}\PY{o}{=}\PY{n}{AC\PYZus{}c\PYZus{}join}\PY{p}{[}\PY{o}{\PYZhy{}}\PY{l+m+mi}{1}\PY{p}{]}\PY{p}{,} \PY{n}{fit\PYZus{}reg}\PY{o}{=}\PY{n+nb+bp}{False}\PY{p}{)} \PY{c}{\PYZsh{} with antithetic correction}
%        \PY{n}{plt}\PY{o}{.}\PY{n}{ylabel}\PY{p}{(}\PY{l+s}{\PYZsq{}}\PY{l+s}{V(T)}\PY{l+s}{\PYZsq{}}\PY{p}{)}
%        \PY{n}{plt}\PY{o}{.}\PY{n}{xlabel}\PY{p}{(}\PY{l+s}{\PYZsq{}}\PY{l+s}{S(T)}\PY{l+s}{\PYZsq{}}\PY{p}{)}
%        \PY{n}{plt}\PY{o}{.}\PY{n}{legend}\PY{p}{(}\PY{p}{[}\PY{l+s}{\PYZsq{}}\PY{l+s}{With Antith. Red.}\PY{l+s}{\PYZsq{}}\PY{p}{]}\PY{p}{,} \PY{n}{loc}\PY{o}{=}\PY{l+s}{\PYZsq{}}\PY{l+s}{upper left}\PY{l+s}{\PYZsq{}}\PY{p}{,} \PY{n}{fontsize}\PY{o}{=}\PY{l+s}{\PYZsq{}}\PY{l+s}{large}\PY{l+s}{\PYZsq{}}\PY{p}{)}
%\end{Verbatim}
%
%            \begin{Verbatim}[commandchars=\\\{\}]
%{\color{outcolor}Out[{\color{outcolor}8}]:} <matplotlib.legend.Legend at 0xaa4e780>
%\end{Verbatim}
        
    \begin{center}
    \adjustimage{max size={0.9\linewidth}{0.9\paperheight}}{asianOptionPricing_files/asianOptionPricing_17_1.png}
    \end{center}
    { \hspace*{\fill} \\}
    
    \subsection{Results and Errors}\label{results-and-errors}

The results from varying the parameters in the model to study the
effects on the pricing estimate are presented now. The key for all below tables
is the following:

\begin{itemize}
\tightlist
\item
  N\_S: number of simulations 
\item
  V: European call equivalent
\item
  AC\_c: Arithmetic continuous average Asian Call price
\item
  AC\_d: Arithmetic discrete average Asian Call price
\item
  GC\_c: Geometric continuous average Asian Call price
\item
  The error is represented in brackets next to the option values and is
  defined as the standard deviation of the estimate divided by the
  square root of the number of simulations (\(\mathrm{std}/ \sqrt{N}\))
\item
  All the option values and errors are displayed to 6 significant
  figures
\end{itemize}

    \subsubsection{Varying the Number of
Simulations}\label{varying-the-number-of-simulations}

Because we are only simulating a finite number of an infinite number of possible stock paths, the error due to using $N$ realisations is $O(N^{1/2})$.
We see that as the number of simulations increases, the error in the
estimate decreases (approximately by a factor of 10 every time the
simulations increase by a factor of 10). Also note that the effect of
the variance reduction technique is to effectively increase N\_S by a
factor of 2, which is why the error presented is smaller than for the
N\_S values shown. Run times weren't compared since the underlying
python code for any of these cases takes less than 2 seconds and this
was deemed quite acceptable. However we expect this to increase  for improved accuracy (as the number of simulations and time steps increase).

%    \begin{Verbatim}[commandchars=\\\{\}]
%{\color{incolor}In [{\color{incolor}9}]:} \PY{n}{df} \PY{o}{=} \PY{n}{pd}\PY{o}{.}\PY{n}{DataFrame}\PY{p}{(}\PY{n}{index}\PY{o}{=}\PY{n+nb}{range}\PY{p}{(}\PY{l+m+mi}{0}\PY{p}{,}\PY{l+m+mi}{3}\PY{p}{)}\PY{p}{)}
%        
%        \PY{c}{\PYZsh{} 100 simulations}
%        \PY{n}{fill\PYZus{}in\PYZus{}df}\PY{p}{(}\PY{n}{df}\PY{p}{,} \PY{n}{dic}\PY{p}{,} \PY{l+m+mi}{0}\PY{p}{)}
%        
%        \PY{c}{\PYZsh{} 1,000 simulations}
%        \PY{n}{dic\PYZus{}1k} \PY{o}{=} \PY{n}{asian\PYZus{}option\PYZus{}simulator}\PY{p}{(}\PY{n}{S0}\PY{o}{=}\PY{l+m+mf}{100.}\PY{p}{,} \PY{n}{K}\PY{o}{=}\PY{l+m+mf}{100.}\PY{p}{,} \PY{n}{T}\PY{o}{=}\PY{l+m+mf}{1.0}\PY{p}{,} \PY{n}{r}\PY{o}{=}\PY{l+m+mf}{0.05}\PY{p}{,} \PY{n}{sigma}\PY{o}{=}\PY{l+m+mf}{0.2}\PY{p}{,} \PY{n}{M}\PY{o}{=}\PY{l+m+mi}{100}\PY{p}{,} \PY{n}{I}\PY{o}{=}\PY{l+m+mi}{1000}\PY{p}{,} \PY{n}{k}\PY{o}{=}\PY{l+m+mi}{10}\PY{p}{)}
%        \PY{n}{fill\PYZus{}in\PYZus{}df}\PY{p}{(}\PY{n}{df}\PY{p}{,} \PY{n}{dic\PYZus{}1k}\PY{p}{,} \PY{l+m+mi}{1}\PY{p}{)}
%        
%        \PY{c}{\PYZsh{} 10,000 simulations}
%        \PY{n}{dic\PYZus{}10k} \PY{o}{=} \PY{n}{asian\PYZus{}option\PYZus{}simulator}\PY{p}{(}\PY{n}{S0}\PY{o}{=}\PY{l+m+mf}{100.}\PY{p}{,} \PY{n}{K}\PY{o}{=}\PY{l+m+mf}{100.}\PY{p}{,} \PY{n}{T}\PY{o}{=}\PY{l+m+mf}{1.0}\PY{p}{,} \PY{n}{r}\PY{o}{=}\PY{l+m+mf}{0.05}\PY{p}{,} \PY{n}{sigma}\PY{o}{=}\PY{l+m+mf}{0.2}\PY{p}{,} \PY{n}{M}\PY{o}{=}\PY{l+m+mi}{100}\PY{p}{,} \PY{n}{I}\PY{o}{=}\PY{l+m+mi}{10000}\PY{p}{,} \PY{n}{k}\PY{o}{=}\PY{l+m+mi}{10}\PY{p}{)}
%        \PY{n}{fill\PYZus{}in\PYZus{}df}\PY{p}{(}\PY{n}{df}\PY{p}{,} \PY{n}{dic\PYZus{}10k}\PY{p}{,} \PY{l+m+mi}{2}\PY{p}{)}
%        
%        \PY{n}{df}\PY{p}{[}\PY{p}{[}\PY{l+s}{\PYZsq{}}\PY{l+s}{N\PYZus{}S}\PY{l+s}{\PYZsq{}}\PY{p}{,} \PY{l+s}{\PYZsq{}}\PY{l+s}{r}\PY{l+s}{\PYZsq{}}\PY{p}{,} \PY{l+s}{\PYZsq{}}\PY{l+s}{sigma}\PY{l+s}{\PYZsq{}}\PY{p}{,} \PY{l+s}{\PYZsq{}}\PY{l+s}{S0}\PY{l+s}{\PYZsq{}}\PY{p}{,} \PY{l+s}{\PYZsq{}}\PY{l+s}{K}\PY{l+s}{\PYZsq{}}\PY{p}{,} \PY{l+s}{\PYZsq{}}\PY{l+s}{Class}\PY{l+s}{\PYZsq{}}\PY{p}{,} \PY{l+s}{\PYZsq{}}\PY{l+s}{V}\PY{l+s}{\PYZsq{}}\PY{p}{,} \PY{l+s}{\PYZsq{}}\PY{l+s}{AC\PYZus{}c}\PY{l+s}{\PYZsq{}}\PY{p}{,} \PY{l+s}{\PYZsq{}}\PY{l+s}{AC\PYZus{}d}\PY{l+s}{\PYZsq{}}\PY{p}{,} \PY{l+s}{\PYZsq{}}\PY{l+s}{GC\PYZus{}c}\PY{l+s}{\PYZsq{}}\PY{p}{]}\PY{p}{]}
%\end{Verbatim}

            \begin{Verbatim}[commandchars=\\\{\}]
{\color{outcolor}Out[{\color{outcolor}9}]:}        N\_S     r  sigma     S0      K Class                     V  \textbackslash{}
        0    100.0  0.05    0.2  100.0  100.0   ATM  10.089201 (0.131768)   
        1   1000.0  0.05    0.2  100.0  100.0   ATM  10.610881 (0.015165)   
        2  10000.0  0.05    0.2  100.0  100.0   ATM  10.388515 (0.001471)   
        
                          AC\_c                 AC\_d                 GC\_c  
        0  5.546591 (0.071872)  5.499740 (0.071726)  5.333837 (0.069442)  
        1  5.783801 (0.008124)  5.707166 (0.008020)  5.564119 (0.007840)  
        2  5.771337 (0.000798)  5.685983 (0.000786)  5.552812 (0.000770)  
\end{Verbatim}
        
    \subsubsection{Varying the Strike}\label{varying-the-strike}

In this case we see the following:

\begin{itemize}
\tightlist
\item
  ITM option is of course more expensive than OTM option for both the
  Asian and European Call
\item
  The difference between the stock price (S0) and exercise price (K) has
  little effect on the estimate error (although there is perhaps a
  slight increase as the option evolves from OTM to ITM)
\item
  The methodology used to calculate the average of the stock (continous
  vs dicrete, arithmetic vs geometric) for the Asian option seems to
  have little effect in the error too
\item
  The price of the European equivalent is always higher and by a
  considerable amount. However the difference (diff\_GC\_c) seems similar
  across ITM, ATM and OTM (although this might not always hold
  depending on other parameters values)
\end{itemize}

%    \begin{Verbatim}[commandchars=\\\{\}]
%{\color{incolor}In [{\color{incolor}10}]:} \PY{n}{df2} \PY{o}{=} \PY{n}{pd}\PY{o}{.}\PY{n}{DataFrame}\PY{p}{(}\PY{n}{index}\PY{o}{=}\PY{n+nb}{range}\PY{p}{(}\PY{l+m+mi}{0}\PY{p}{,}\PY{l+m+mi}{3}\PY{p}{)}\PY{p}{)}
%         
%         \PY{c}{\PYZsh{} OTM, 10k simulations}
%         \PY{n}{dic\PYZus{}OTM} \PY{o}{=} \PY{n}{asian\PYZus{}option\PYZus{}simulator}\PY{p}{(}\PY{n}{S0}\PY{o}{=}\PY{l+m+mf}{100.}\PY{p}{,} \PY{n}{K}\PY{o}{=}\PY{l+m+mf}{95.}\PY{p}{,} \PY{n}{T}\PY{o}{=}\PY{l+m+mf}{1.0}\PY{p}{,} \PY{n}{r}\PY{o}{=}\PY{l+m+mf}{0.05}\PY{p}{,} \PY{n}{sigma}\PY{o}{=}\PY{l+m+mf}{0.2}\PY{p}{,} \PY{n}{M}\PY{o}{=}\PY{l+m+mi}{100}\PY{p}{,} \PY{n}{I}\PY{o}{=}\PY{l+m+mi}{10000}\PY{p}{,} \PY{n}{k}\PY{o}{=}\PY{l+m+mi}{10}\PY{p}{)}
%         \PY{n}{fill\PYZus{}in\PYZus{}df}\PY{p}{(}\PY{n}{df2}\PY{p}{,} \PY{n}{dic\PYZus{}OTM}\PY{p}{,} \PY{l+m+mi}{0}\PY{p}{)}
%         
%         \PY{c}{\PYZsh{} ATM, 10k simulations}
%         \PY{n}{df2}\PY{o}{.}\PY{n}{loc}\PY{p}{[}\PY{l+m+mi}{1}\PY{p}{]} \PY{o}{=} \PY{n}{df}\PY{o}{.}\PY{n}{loc}\PY{p}{[}\PY{l+m+mi}{2}\PY{p}{]}
%         
%         \PY{c}{\PYZsh{} ITM, 10k simulations}
%         \PY{n}{dic\PYZus{}ITM} \PY{o}{=} \PY{n}{asian\PYZus{}option\PYZus{}simulator}\PY{p}{(}\PY{n}{S0}\PY{o}{=}\PY{l+m+mf}{100.}\PY{p}{,} \PY{n}{K}\PY{o}{=}\PY{l+m+mf}{105.}\PY{p}{,} \PY{n}{T}\PY{o}{=}\PY{l+m+mf}{1.0}\PY{p}{,} \PY{n}{r}\PY{o}{=}\PY{l+m+mf}{0.05}\PY{p}{,} \PY{n}{sigma}\PY{o}{=}\PY{l+m+mf}{0.2}\PY{p}{,} \PY{n}{M}\PY{o}{=}\PY{l+m+mi}{100}\PY{p}{,} \PY{n}{I}\PY{o}{=}\PY{l+m+mi}{10000}\PY{p}{,} \PY{n}{k}\PY{o}{=}\PY{l+m+mi}{10}\PY{p}{)}
%         \PY{n}{fill\PYZus{}in\PYZus{}df}\PY{p}{(}\PY{n}{df2}\PY{p}{,} \PY{n}{dic\PYZus{}ITM}\PY{p}{,} \PY{l+m+mi}{2}\PY{p}{)}
%         
%         \PY{n}{df2}\PY{p}{[}\PY{p}{[}\PY{l+s}{\PYZsq{}}\PY{l+s}{N\PYZus{}S}\PY{l+s}{\PYZsq{}}\PY{p}{,} \PY{l+s}{\PYZsq{}}\PY{l+s}{r}\PY{l+s}{\PYZsq{}}\PY{p}{,} \PY{l+s}{\PYZsq{}}\PY{l+s}{sigma}\PY{l+s}{\PYZsq{}}\PY{p}{,} \PY{l+s}{\PYZsq{}}\PY{l+s}{S0}\PY{l+s}{\PYZsq{}}\PY{p}{,} \PY{l+s}{\PYZsq{}}\PY{l+s}{K}\PY{l+s}{\PYZsq{}}\PY{p}{,} \PY{l+s}{\PYZsq{}}\PY{l+s}{Class}\PY{l+s}{\PYZsq{}}\PY{p}{,} \PY{l+s}{\PYZsq{}}\PY{l+s}{V}\PY{l+s}{\PYZsq{}}\PY{p}{,} \PY{l+s}{\PYZsq{}}\PY{l+s}{AC\PYZus{}c}\PY{l+s}{\PYZsq{}}\PY{p}{,} \PY{l+s}{\PYZsq{}}\PY{l+s}{AC\PYZus{}d}\PY{l+s}{\PYZsq{}}\PY{p}{,} \PY{l+s}{\PYZsq{}}\PY{l+s}{GC\PYZus{}c}\PY{l+s}{\PYZsq{}}\PY{p}{,} \PY{l+s}{\PYZsq{}}\PY{l+s}{diff\PYZus{}GC\PYZus{}c}\PY{l+s}{\PYZsq{}}\PY{p}{]}\PY{p}{]}
%\end{Verbatim}

            \begin{Verbatim}[commandchars=\\\{\}]
{\color{outcolor}Out[{\color{outcolor}10}]:}        N\_S     r  sigma     S0      K Class                     V  \textbackslash{}
         0  10000.0  0.05    0.2  100.0   95.0   ITM  13.306567 (0.001611)   
         1  10000.0  0.05    0.2  100.0  100.0   ATM  10.388515 (0.001471)   
         2  10000.0  0.05    0.2  100.0  105.0   OTM   7.957399 (0.001319)   
         
                           AC\_c                 AC\_d                 GC\_c diff\_GC\_c  
         0  8.825431 (0.000938)  8.753649 (0.000925)  8.575778 (0.000912)  4.730789  
         1  5.771337 (0.000798)  5.685983 (0.000786)  5.552812 (0.000770)  4.835703  
         2  3.514907 (0.000641)  3.432210 (0.000630)  3.330691 (0.000612)  4.626708  
\end{Verbatim}
        
    \subsubsection{Varying the Volatility}\label{varying-the-volatility}

The error appears to significantly increase when the volatility of the
stock increases. This is expected since an increase in uncertainty in
stock prices undermines the predictability of the model itself. The
increase in option price with volatility is also readily confirmed.

%    \begin{Verbatim}[commandchars=\\\{\}]
%{\color{incolor}In [{\color{incolor}11}]:} \PY{n}{df3} \PY{o}{=} \PY{n}{pd}\PY{o}{.}\PY{n}{DataFrame}\PY{p}{(}\PY{n}{index}\PY{o}{=}\PY{n+nb}{range}\PY{p}{(}\PY{l+m+mi}{0}\PY{p}{,}\PY{l+m+mi}{3}\PY{p}{)}\PY{p}{)}
%         
%         \PY{c}{\PYZsh{} Vol=20\PYZpc{}}
%         \PY{n}{dic\PYZus{}vol30} \PY{o}{=} \PY{n}{asian\PYZus{}option\PYZus{}simulator}\PY{p}{(}\PY{n}{S0}\PY{o}{=}\PY{l+m+mf}{100.}\PY{p}{,} \PY{n}{K}\PY{o}{=}\PY{l+m+mf}{100.}\PY{p}{,} \PY{n}{T}\PY{o}{=}\PY{l+m+mf}{1.0}\PY{p}{,} \PY{n}{r}\PY{o}{=}\PY{l+m+mf}{0.05}\PY{p}{,} \PY{n}{sigma}\PY{o}{=}\PY{l+m+mf}{0.2}\PY{p}{,} \PY{n}{M}\PY{o}{=}\PY{l+m+mi}{100}\PY{p}{,} \PY{n}{I}\PY{o}{=}\PY{l+m+mi}{10000}\PY{p}{,} \PY{n}{k}\PY{o}{=}\PY{l+m+mi}{10}\PY{p}{)}
%         \PY{n}{fill\PYZus{}in\PYZus{}df}\PY{p}{(}\PY{n}{df3}\PY{p}{,} \PY{n}{dic\PYZus{}vol30}\PY{p}{,} \PY{l+m+mi}{0}\PY{p}{)}
%         
%         \PY{c}{\PYZsh{} Vol=30\PYZpc{}}
%         \PY{n}{dic\PYZus{}vol30} \PY{o}{=} \PY{n}{asian\PYZus{}option\PYZus{}simulator}\PY{p}{(}\PY{n}{S0}\PY{o}{=}\PY{l+m+mf}{100.}\PY{p}{,} \PY{n}{K}\PY{o}{=}\PY{l+m+mf}{100.}\PY{p}{,} \PY{n}{T}\PY{o}{=}\PY{l+m+mf}{1.0}\PY{p}{,} \PY{n}{r}\PY{o}{=}\PY{l+m+mf}{0.05}\PY{p}{,} \PY{n}{sigma}\PY{o}{=}\PY{l+m+mf}{0.3}\PY{p}{,} \PY{n}{M}\PY{o}{=}\PY{l+m+mi}{100}\PY{p}{,} \PY{n}{I}\PY{o}{=}\PY{l+m+mi}{10000}\PY{p}{,} \PY{n}{k}\PY{o}{=}\PY{l+m+mi}{10}\PY{p}{)}
%         \PY{n}{fill\PYZus{}in\PYZus{}df}\PY{p}{(}\PY{n}{df3}\PY{p}{,} \PY{n}{dic\PYZus{}vol30}\PY{p}{,} \PY{l+m+mi}{1}\PY{p}{)}
%         
%         \PY{c}{\PYZsh{} Vol=40\PYZpc{}}
%         \PY{n}{dic\PYZus{}vol40} \PY{o}{=} \PY{n}{asian\PYZus{}option\PYZus{}simulator}\PY{p}{(}\PY{n}{S0}\PY{o}{=}\PY{l+m+mf}{100.}\PY{p}{,} \PY{n}{K}\PY{o}{=}\PY{l+m+mf}{100.}\PY{p}{,} \PY{n}{T}\PY{o}{=}\PY{l+m+mf}{1.0}\PY{p}{,} \PY{n}{r}\PY{o}{=}\PY{l+m+mf}{0.05}\PY{p}{,} \PY{n}{sigma}\PY{o}{=}\PY{l+m+mf}{0.4}\PY{p}{,} \PY{n}{M}\PY{o}{=}\PY{l+m+mi}{100}\PY{p}{,} \PY{n}{I}\PY{o}{=}\PY{l+m+mi}{10000}\PY{p}{,} \PY{n}{k}\PY{o}{=}\PY{l+m+mi}{10}\PY{p}{)}
%         \PY{n}{fill\PYZus{}in\PYZus{}df}\PY{p}{(}\PY{n}{df3}\PY{p}{,} \PY{n}{dic\PYZus{}vol40}\PY{p}{,} \PY{l+m+mi}{2}\PY{p}{)}
%         
%         \PY{n}{df3}\PY{p}{[}\PY{p}{[}\PY{l+s}{\PYZsq{}}\PY{l+s}{N\PYZus{}S}\PY{l+s}{\PYZsq{}}\PY{p}{,} \PY{l+s}{\PYZsq{}}\PY{l+s}{r}\PY{l+s}{\PYZsq{}}\PY{p}{,} \PY{l+s}{\PYZsq{}}\PY{l+s}{sigma}\PY{l+s}{\PYZsq{}}\PY{p}{,} \PY{l+s}{\PYZsq{}}\PY{l+s}{S0}\PY{l+s}{\PYZsq{}}\PY{p}{,} \PY{l+s}{\PYZsq{}}\PY{l+s}{K}\PY{l+s}{\PYZsq{}}\PY{p}{,} \PY{l+s}{\PYZsq{}}\PY{l+s}{Class}\PY{l+s}{\PYZsq{}}\PY{p}{,} \PY{l+s}{\PYZsq{}}\PY{l+s}{V}\PY{l+s}{\PYZsq{}}\PY{p}{,} \PY{l+s}{\PYZsq{}}\PY{l+s}{AC\PYZus{}c}\PY{l+s}{\PYZsq{}}\PY{p}{,} \PY{l+s}{\PYZsq{}}\PY{l+s}{AC\PYZus{}d}\PY{l+s}{\PYZsq{}}\PY{p}{,} \PY{l+s}{\PYZsq{}}\PY{l+s}{GC\PYZus{}c}\PY{l+s}{\PYZsq{}}\PY{p}{]}\PY{p}{]}
%\end{Verbatim}

            \begin{Verbatim}[commandchars=\\\{\}]
{\color{outcolor}Out[{\color{outcolor}11}]:}        N\_S     r  sigma     S0      K Class                     V  \textbackslash{}
         0  10000.0  0.05    0.2  100.0  100.0   ATM  10.388515 (0.001471)   
         1  10000.0  0.05    0.3  100.0  100.0   ATM  14.135187 (0.002251)   
         2  10000.0  0.05    0.4  100.0  100.0   ATM  17.893775 (0.003127)   
         
                            AC\_c                 AC\_d                 GC\_c  
         0   5.771337 (0.000798)  5.685983 (0.000786)  5.552812 (0.000770)  
         1   7.961140 (0.001204)  7.832518 (0.001187)  7.507157 (0.001140)  
         2  10.158977 (0.001646)  9.986547 (0.001624)  9.379621 (0.001526)  
\end{Verbatim}
        
    \subsubsection{Varying the interest
rate}\label{varying-the-interest-rate}

In contrast to volatility, varying the interest rate has a negligible
effect on the error. The increase in option value seems to be slower
than that observed with volatility variation.

%    \begin{Verbatim}[commandchars=\\\{\}]
%{\color{incolor}In [{\color{incolor}12}]:} \PY{n}{df4} \PY{o}{=} \PY{n}{pd}\PY{o}{.}\PY{n}{DataFrame}\PY{p}{(}\PY{n}{index}\PY{o}{=}\PY{n+nb}{range}\PY{p}{(}\PY{l+m+mi}{0}\PY{p}{,}\PY{l+m+mi}{3}\PY{p}{)}\PY{p}{)}
%         
%         \PY{c}{\PYZsh{} r=3\PYZpc{}}
%         \PY{n}{dic\PYZus{}r3} \PY{o}{=} \PY{n}{asian\PYZus{}option\PYZus{}simulator}\PY{p}{(}\PY{n}{S0}\PY{o}{=}\PY{l+m+mf}{100.}\PY{p}{,} \PY{n}{K}\PY{o}{=}\PY{l+m+mf}{100.}\PY{p}{,} \PY{n}{T}\PY{o}{=}\PY{l+m+mf}{1.0}\PY{p}{,} \PY{n}{r}\PY{o}{=}\PY{l+m+mf}{0.03}\PY{p}{,} \PY{n}{sigma}\PY{o}{=}\PY{l+m+mf}{0.2}\PY{p}{,} \PY{n}{M}\PY{o}{=}\PY{l+m+mi}{100}\PY{p}{,} \PY{n}{I}\PY{o}{=}\PY{l+m+mi}{10000}\PY{p}{,} \PY{n}{k}\PY{o}{=}\PY{l+m+mi}{10}\PY{p}{)}
%         \PY{n}{fill\PYZus{}in\PYZus{}df}\PY{p}{(}\PY{n}{df4}\PY{p}{,} \PY{n}{dic\PYZus{}r3}\PY{p}{,} \PY{l+m+mi}{0}\PY{p}{)}
%         
%         \PY{c}{\PYZsh{} r=5\PYZpc{}}
%         \PY{n}{dic\PYZus{}r5} \PY{o}{=} \PY{n}{asian\PYZus{}option\PYZus{}simulator}\PY{p}{(}\PY{n}{S0}\PY{o}{=}\PY{l+m+mf}{100.}\PY{p}{,} \PY{n}{K}\PY{o}{=}\PY{l+m+mf}{100.}\PY{p}{,} \PY{n}{T}\PY{o}{=}\PY{l+m+mf}{1.0}\PY{p}{,} \PY{n}{r}\PY{o}{=}\PY{l+m+mf}{0.05}\PY{p}{,} \PY{n}{sigma}\PY{o}{=}\PY{l+m+mf}{0.2}\PY{p}{,} \PY{n}{M}\PY{o}{=}\PY{l+m+mi}{100}\PY{p}{,} \PY{n}{I}\PY{o}{=}\PY{l+m+mi}{10000}\PY{p}{,} \PY{n}{k}\PY{o}{=}\PY{l+m+mi}{10}\PY{p}{)}
%         \PY{n}{fill\PYZus{}in\PYZus{}df}\PY{p}{(}\PY{n}{df4}\PY{p}{,} \PY{n}{dic\PYZus{}r5}\PY{p}{,} \PY{l+m+mi}{1}\PY{p}{)}
%         
%         \PY{c}{\PYZsh{} r=7\PYZpc{}}
%         \PY{n}{dic\PYZus{}r7} \PY{o}{=} \PY{n}{asian\PYZus{}option\PYZus{}simulator}\PY{p}{(}\PY{n}{S0}\PY{o}{=}\PY{l+m+mf}{100.}\PY{p}{,} \PY{n}{K}\PY{o}{=}\PY{l+m+mf}{100.}\PY{p}{,} \PY{n}{T}\PY{o}{=}\PY{l+m+mf}{1.0}\PY{p}{,} \PY{n}{r}\PY{o}{=}\PY{l+m+mf}{0.07}\PY{p}{,} \PY{n}{sigma}\PY{o}{=}\PY{l+m+mf}{0.2}\PY{p}{,} \PY{n}{M}\PY{o}{=}\PY{l+m+mi}{100}\PY{p}{,} \PY{n}{I}\PY{o}{=}\PY{l+m+mi}{10000}\PY{p}{,} \PY{n}{k}\PY{o}{=}\PY{l+m+mi}{10}\PY{p}{)}
%         \PY{n}{fill\PYZus{}in\PYZus{}df}\PY{p}{(}\PY{n}{df4}\PY{p}{,} \PY{n}{dic\PYZus{}r7}\PY{p}{,} \PY{l+m+mi}{2}\PY{p}{)}
%         
%         \PY{n}{df4}\PY{p}{[}\PY{p}{[}\PY{l+s}{\PYZsq{}}\PY{l+s}{N\PYZus{}S}\PY{l+s}{\PYZsq{}}\PY{p}{,} \PY{l+s}{\PYZsq{}}\PY{l+s}{r}\PY{l+s}{\PYZsq{}}\PY{p}{,} \PY{l+s}{\PYZsq{}}\PY{l+s}{sigma}\PY{l+s}{\PYZsq{}}\PY{p}{,} \PY{l+s}{\PYZsq{}}\PY{l+s}{S0}\PY{l+s}{\PYZsq{}}\PY{p}{,} \PY{l+s}{\PYZsq{}}\PY{l+s}{K}\PY{l+s}{\PYZsq{}}\PY{p}{,} \PY{l+s}{\PYZsq{}}\PY{l+s}{Class}\PY{l+s}{\PYZsq{}}\PY{p}{,} \PY{l+s}{\PYZsq{}}\PY{l+s}{V}\PY{l+s}{\PYZsq{}}\PY{p}{,} \PY{l+s}{\PYZsq{}}\PY{l+s}{AC\PYZus{}c}\PY{l+s}{\PYZsq{}}\PY{p}{,} \PY{l+s}{\PYZsq{}}\PY{l+s}{AC\PYZus{}d}\PY{l+s}{\PYZsq{}}\PY{p}{,} \PY{l+s}{\PYZsq{}}\PY{l+s}{GC\PYZus{}c}\PY{l+s}{\PYZsq{}}\PY{p}{]}\PY{p}{]}
%\end{Verbatim}

            \begin{Verbatim}[commandchars=\\\{\}]
{\color{outcolor}Out[{\color{outcolor}12}]:}        N\_S     r  sigma     S0      K Class                     V  \textbackslash{}
         0  10000.0  0.03    0.2  100.0  100.0   ATM   9.352018 (0.001411)   
         1  10000.0  0.05    0.2  100.0  100.0   ATM  10.388515 (0.001471)   
         2  10000.0  0.07    0.2  100.0  100.0   ATM  11.485530 (0.001528)   
         
                           AC\_c                 AC\_d                 GC\_c  
         0  5.293044 (0.000773)  5.204118 (0.000761)  5.094293 (0.000747)  
         1  5.771337 (0.000798)  5.685983 (0.000786)  5.552812 (0.000770)  
         2  6.268640 (0.000822)  6.188231 (0.000811)  6.028783 (0.000792)  
\end{Verbatim}
        
    \subsubsection{Float vs Fixed Strike}\label{float-vs-fixed-strike}

For the Asian call option the float strike was defined here as:

\begin{equation}
C_{float}(T) = \text{max}\left( S(T) -  A(0,T), 0 \right)
\end{equation}
i.e. the strike is taken to be the value of the stock at expiry.

Within errors, the float and fixed strike Asian options seem to be
nearly the same, suggesting an `equivalence' relationship between them.

%    \begin{Verbatim}[commandchars=\\\{\}]
%{\color{incolor}In [{\color{incolor}13}]:} \PY{n}{df5} \PY{o}{=} \PY{n}{pd}\PY{o}{.}\PY{n}{DataFrame}\PY{p}{(}\PY{n}{index}\PY{o}{=}\PY{n+nb}{range}\PY{p}{(}\PY{l+m+mi}{0}\PY{p}{,}\PY{l+m+mi}{2}\PY{p}{)}\PY{p}{)}
%         
%         \PY{c}{\PYZsh{} Fixed (default mode)}
%         \PY{n}{dic\PYZus{}fx} \PY{o}{=} \PY{n}{asian\PYZus{}option\PYZus{}simulator}\PY{p}{(}\PY{n}{S0}\PY{o}{=}\PY{l+m+mf}{100.}\PY{p}{,} \PY{n}{K}\PY{o}{=}\PY{l+m+mf}{100.}\PY{p}{,} \PY{n}{T}\PY{o}{=}\PY{l+m+mf}{1.0}\PY{p}{,} \PY{n}{r}\PY{o}{=}\PY{l+m+mf}{0.05}\PY{p}{,} \PY{n}{sigma}\PY{o}{=}\PY{l+m+mf}{0.2}\PY{p}{,} \PY{n}{M}\PY{o}{=}\PY{l+m+mi}{100}\PY{p}{,} \PY{n}{I}\PY{o}{=}\PY{l+m+mi}{10000}\PY{p}{,} \PY{n}{k}\PY{o}{=}\PY{l+m+mi}{10}\PY{p}{)}
%         \PY{n}{fill\PYZus{}in\PYZus{}df}\PY{p}{(}\PY{n}{df5}\PY{p}{,} \PY{n}{dic\PYZus{}fx}\PY{p}{,} \PY{l+m+mi}{0}\PY{p}{)}
%         \PY{c}{\PYZsh{} Float}
%         \PY{n}{dic\PYZus{}fl} \PY{o}{=} \PY{n}{asian\PYZus{}option\PYZus{}simulator}\PY{p}{(}\PY{n}{S0}\PY{o}{=}\PY{l+m+mf}{100.}\PY{p}{,} \PY{n}{K}\PY{o}{=}\PY{l+m+mf}{100.}\PY{p}{,} \PY{n}{T}\PY{o}{=}\PY{l+m+mf}{1.0}\PY{p}{,} \PY{n}{r}\PY{o}{=}\PY{l+m+mf}{0.05}\PY{p}{,} \PY{n}{sigma}\PY{o}{=}\PY{l+m+mf}{0.2}\PY{p}{,} \PY{n}{M}\PY{o}{=}\PY{l+m+mi}{100}\PY{p}{,} \PY{n}{I}\PY{o}{=}\PY{l+m+mi}{10000}\PY{p}{,} \PY{n}{k}\PY{o}{=}\PY{l+m+mi}{10}\PY{p}{,} \PY{n}{mode}\PY{o}{=}\PY{l+s}{\PYZsq{}}\PY{l+s}{float}\PY{l+s}{\PYZsq{}}\PY{p}{)}
%         \PY{n}{fill\PYZus{}in\PYZus{}df}\PY{p}{(}\PY{n}{df5}\PY{p}{,} \PY{n}{dic\PYZus{}fl}\PY{p}{,} \PY{l+m+mi}{1}\PY{p}{)}
%         
%         \PY{n}{df5}\PY{p}{[}\PY{p}{[}\PY{l+s}{\PYZsq{}}\PY{l+s}{N\PYZus{}S}\PY{l+s}{\PYZsq{}}\PY{p}{,} \PY{l+s}{\PYZsq{}}\PY{l+s}{mode}\PY{l+s}{\PYZsq{}}\PY{p}{,} \PY{l+s}{\PYZsq{}}\PY{l+s}{r}\PY{l+s}{\PYZsq{}}\PY{p}{,} \PY{l+s}{\PYZsq{}}\PY{l+s}{sigma}\PY{l+s}{\PYZsq{}}\PY{p}{,} \PY{l+s}{\PYZsq{}}\PY{l+s}{S0}\PY{l+s}{\PYZsq{}}\PY{p}{,} \PY{l+s}{\PYZsq{}}\PY{l+s}{AC\PYZus{}c}\PY{l+s}{\PYZsq{}}\PY{p}{,} \PY{l+s}{\PYZsq{}}\PY{l+s}{AC\PYZus{}d}\PY{l+s}{\PYZsq{}}\PY{p}{,} \PY{l+s}{\PYZsq{}}\PY{l+s}{GC\PYZus{}c}\PY{l+s}{\PYZsq{}}\PY{p}{]}\PY{p}{]}
%\end{Verbatim}

            \begin{Verbatim}[commandchars=\\\{\}]
{\color{outcolor}Out[{\color{outcolor}13}]:}        N\_S   mode     r  sigma     S0                 AC\_c  \textbackslash{}
         0  10000.0  fixed  0.05    0.2  100.0  5.771337 (0.000798)   
         1  10000.0  float  0.05    0.2  100.0  5.808603 (0.000846)   
         
                           AC\_d                 GC\_c  
         0  5.685983 (0.000786)  5.552812 (0.000770)  
         1  5.715516 (0.000831)  6.019974 (0.000874)  
\end{Verbatim}
        
    \subsection{Summary}\label{summary}

In summary, Monte Carlo seems a suitable method to efficiently price European
and Asian options. The number time steps and simulations have the biggest effect on
the option price accuracy, followed by volatility. The variance
reduction technique proved to be useful to reduce significantly the
error at little computing cost. An equivalence between float and fixed
strikes in the pricing seems to suggest an equivalence relationship, although to
validate this, more test cases should be examined. For all test cases,
the Asian option price appears considerably lower than the European,
making it an attractive financial instrument for certain risk management
strategies.


    % Add a bibliography block to the postdoc
    
    
    
    \end{document}
